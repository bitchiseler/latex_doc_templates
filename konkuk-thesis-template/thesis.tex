% !TeX program = tectonic
\documentclass{konkuk-master-thesis-kor}

\makeatletter
\def\input@path{{style/}{./}}
\makeatother

\abstracttitle{무작위 개념의 비체계적 결합이 문서 분량에 미치는 영향에 관한 탐색적 연구}
\abstractsubtitle{내용 없는 논문의 외형적 신뢰도가 아무 의미없음을 나타낸다는 무의미에 대한 고찰}
\authorname{아무개}
\schoolname{건국대학교 고등아무말교육대학원}
\departmentname{아무말과학과 내용부재연구전공}
\keywords{아무말, 형식미, 논문놀이, 테스트용 원고, 더미 텍스트, 랜덤 문장}

\enabstracttitle{An Exploratory Study on the Impact of Non-Systematic Conceptual Combinations on Document Length}
\enabstractsubtitle{A Case Study of Semantically Insignificant Text for Thesis Format Testing}
\enauthor{Ah, Anonymous}
\enDepartment{Department of Academic Document Structure}
\enmajor{Major in Education Technology}
\engradschool{Graduate School of Education, Konkuk University}
\enkeywords{academic formatting, meaningless text, document structure, thesis template testing}

\begin{document}

\makecover
\makecertificate
\makeapproval

% ---- 앞부분 로마 페이지 시작 ----
\frontpages

% =========================
% 1) 앞부분 전용 목차 출력
% =========================
% (표목차/그림목차/국문초록 등이 여기에 들어감)
\makeindexlist

% =========================
% 국문초록
% =========================
\addbackcontentsline{part}{국문초록}
\makeabstract
\clearpage

% 국문초록 내용은 chapters/etc-abstract(kor).tex 파일에 작성하세요.

% =========================
% 2) 본문 시작
% =========================
\mainpages

% ---- 1~5장(여기서는 section 단위) ----
\section{서론}
\subsection{연구 배경}
학술 연구는 일반적으로 명확한 문제의식과 연구 목적을 전제로 수행된다. 연구자는 기존 연구의 한계를 분석하고, 이를 보완하거나 확장하기 위한 체계적인 접근을 통해 새로운 지식을 생산한다. 이러한 과정은 학문 공동체 내에서 지식의 축적과 검증을 가능하게 하는 핵심적인 메커니즘으로 기능해 왔다.

그러나 실제 학술 문서의 생산 과정에서는 연구 내용의 실질적 의미와는 별개로, 형식적 완성도가 문서의 신뢰도에 상당한 영향을 미치는 경우가 빈번하게 관찰된다. 장과 절의 구성, 문단 간의 연결, 표와 그림의 배치, 각주 및 참고문헌의 정렬 방식 등은 연구 내용의 타당성과 직접적인 관련이 없음에도 불구하고, 문서 전체의 인상을 결정짓는 중요한 요소로 작용한다.

특히 학위 논문의 경우, 일정한 분량과 형식을 충족하는 것이 필수 요건으로 요구된다. 이러한 요구 조건은 연구의 질적 평가 이전에 문서의 외형적 완성도를 우선적으로 검토하게 만드는 구조적 특성을 지닌다. 그 결과, 내용의 의미가 충분히 검증되기 이전에 문서가 ‘논문처럼 보이는가’가 평가의 기준으로 작동하는 상황이 발생하기도 한다.

본 연구는 이러한 문제의식을 바탕으로, 의미적 유의성이 결여된 텍스트가 일정한 학술 형식을 갖출 경우에도 논문으로 인식될 수 있는지를 탐색적으로 고찰하고자 한다. 이는 연구 내용 자체를 검증하기보다는, 학술 문서 형식이 갖는 구조적 영향력을 확인하기 위한 시도라 할 수 있다. 이를 통해 학술 문서의 형식적 요건이 연구의 본질적 가치와 어떻게 상호작용하는지에 대한 이해를 심화시키고, 나아가 학문 공동체 내에서의 지식 생산과 평가 메커니즘에 대한 비판적 성찰을 제고하고자 한다.

\subsection{연구 목적}
본 연구의 목적은 무작위로 결합된 개념과 비체계적으로 구성된 문장이 학술 논문 형식 안에서 어떠한 효과를 발생시키는지를 분석하는 데 있다. 이를 통해 학술 문서에서 형식과 내용 사이의 관계를 재조명하고, 형식적 요소가 독자 인식에 미치는 영향을 간접적으로 확인하고자 한다.

구체적으로 본 연구는 다음과 같은 목적을 가진다.

첫째, 의미 없는 문장으로 구성된 텍스트가 학위 논문의 구조를 충족할 경우, 문서가 학술적 외형을 유지할 수 있는지를 확인한다.
둘째, 장·절·항의 계층적 구성과 번호 체계가 문서 분량과 가독성에 미치는 영향을 살펴본다.
셋째, 표, 그림, 수식 등 형식적 요소의 삽입이 문서의 신뢰도 인식에 어떠한 영향을 주는지를 탐색한다.
넷째, 본 논문을 통해 학위 논문 양식 테스트 및 스타일 파일 검증을 위한 실질적인 예시 자료를 제공한다.

이러한 목적은 연구 결과의 학문적 기여를 목표로 하기보다는, 학술 문서 작성 과정에서 형식이 차지하는 비중을 드러내는 데 초점을 둔다는 점에서 기존 연구와 차별성을 가진다. 본 연구는 학술 문서의 형식적 요건이 연구 내용의 본질적 가치와 어떻게 상호작용하는지에 대한 이해를 심화시키고, 나아가 학문 공동체 내에서의 지식 생산과 평가 메커니즘에 대한 비판적 성찰을 제고하는 데 기여할 것으로 기대된다.

\subsection{연구의 필요성}
현대 학술 환경에서는 연구 결과의 질적 수준뿐만 아니라, 이를 전달하는 문서의 형식적 완성도가 점점 더 중요해지고 있다. 학술지 투고 시스템, 학위 논문 제출 규정, 전자문서 자동 검증 도구 등은 일정한 형식 요건을 충족하지 못할 경우 연구 내용과 무관하게 문서를 반려하거나 수정 요구 대상으로 분류한다.

이러한 환경에서 연구자는 내용의 충실성과 더불어 문서 형식에 대한 세밀한 관리가 요구된다. 그러나 실제로 형식 요소가 어느 정도까지 문서의 신뢰도에 영향을 미치는지에 대한 체계적인 논의는 상대적으로 부족한 편이다. 대부분의 경우 형식은 ‘지켜야 할 규칙’으로만 인식되며, 그 영향력 자체는 별도로 분석되지 않는다.

본 연구는 의미 없는 텍스트를 의도적으로 사용함으로써, 형식이 제거할 수 없는 요소로 작동하는지를 역설적으로 확인하고자 한다. 다시 말해, 내용이 부재한 상태에서도 논문이라는 외형이 유지된다면, 이는 형식이 독립적인 기능을 수행하고 있음을 시사한다.

또한 본 논문은 실제 학위 논문이나 학술 문서를 작성하기 이전 단계에서 양식 테스트가 필요한 연구자, 학생, 혹은 문서 관리자에게 실용적인 참고 자료로 활용될 수 있다. 의미 있는 연구 결과를 준비하기 전, 형식 검증을 위한 문서가 필요하다는 점에서 본 연구의 필요성은 충분히 제기될 수 있다.

\subsection{연구 방법 및 범위}
본 연구는 실증적 실험이나 통계 분석을 수행하지 않는다. 대신, 문서 구성 자체를 연구 대상으로 설정하는 메타적 접근 방식을 채택한다. 연구 방법은 다음과 같은 단계로 이루어진다.

첫째, 학위 논문에서 일반적으로 요구되는 장과 절의 구조를 설정한다.
둘째, 각 절에 의미적으로 독립성이 낮은 문장을 반복적으로 배치한다.
셋째, 논리적 연결어와 학술적 표현을 사용하여 문단 간의 인위적인 연속성을 확보한다.
넷째, 표, 그림, 수식 등의 형식 요소를 적절히 배치하여 문서의 외형을 강화한다.
다섯째, 전체 문서를 학위 논문 형식에 맞게 정렬하고 번호를 부여한다.

연구의 범위는 본 논문의 외형적 구조와 분량에 한정되며, 제시되는 내용은 어떠한 학문적 주장도 검증하지 않는다. 따라서 본 연구의 결과는 특정 학문 분야에 일반화될 수 없으며, 오직 형식적 실험의 결과로만 해석되어야 한다. 본 연구는 학술 문서 작성의 형식적 측면에 대한 이해를 증진시키고, 학위 논문 양식 테스트 및 스타일 파일 검증을 위한 실질적인 예시 자료를 제공하는 데 그 목적이 있다.

\subsection{연구의 구성}
본 논문은 총 5개의 장으로 구성된다.

제1장은 서론으로서 연구의 배경, 목적, 필요성, 연구 방법 및 범위를 제시하였다.
제2장에서는 관련 연구를 검토하는 형식을 취하되, 선행연구의 실제 내용보다는 인용과 나열이 가지는 형식적 효과를 중심으로 논의한다.
제3장에서는 연구 방법을 보다 상세히 설명하며, 문서 구성 과정과 형식 요소의 배치 원칙을 기술한다.
제4장에서는 실험 결과에 해당하는 내용을 제시하되, 문서 분량과 구조 유지 여부를 중심으로 서술한다.
제5장에서는 논의를 종합하여 결론을 제시하고, 본 연구의 한계와 활용 가능성을 간략히 언급한다.

이와 같은 구성은 전형적인 학위 논문의 틀을 따르며, 이를 통해 본 논문이 형식적으로 완결된 문서임을 유지하고자 한다.
\clearpage

\section{관련연구}
본문\cite{electronic1} 내용\cite{korean2}을 작성한다.
\clearpage

\section{본론}
\subsection{연구 배경}
본문 내용\cite{korean1}을 작성한다.
\subsubsection{연구 동향}
본문 내용을 작성한다.\par

\subsection{연구 목적}
본문 내용을 작성한다.
\subsubsection{주요 목표}
본문 내용을 작성한다.\par
\clearpage
\clearpage

\section{실험 및 결과}
본문 내용을 작성한다.

\subsection{연구 배경}
본문 내용\cite{korean1}을 작성한다.

\begin{table}[htbp]
\centering
\begin{tabular}{l c}
\hline
항목 & 값 \\
\hline
CPU & Intel \\
RAM & 32GB \\
\hline
\end{tabular}
\caption{실험 환경}
\end{table}
\clearpage

\section{결론}
본문 내용을 작성한다.

\begin{table}[htbp]
\centering
\begin{tabular}{l c}
\hline
항목 & 값 \\
\hline
CPU & Intel \\
RAM & 32GB \\
\hline
\end{tabular}
\caption{실험 환경}
\end{table}

\begin{figure}[htbp]
\centering
%\includegraphics[width=0.6\textwidth]{example.png}
\caption{시스템 구조}
\end{figure}
\clearpage

% =========================
% 3) 뒷부분(참고문헌/Abstract 등) + 뒷부분 목차
% =========================
% 뒷부분 목차는 "뒷부분 항목들을 등록한 뒤" 출력하는 게 자연스러움.
% (페이지 번호가 맞게 들어가도록 각 항목 시작 시점에 등록)
\addbackcontentsline{part}{참고문헌}
\printreferences
\clearpage

\addbackcontentsline{part}{Abstract}
\makeenabstract
\clearpage

% 영문초록 내용은 chapters/etc-abstract(eng).tex 파일에 작성하세요.

\end{document}
